\documentclass[main.tex]{subfiles}

\begin{document}

\section{Future Work}
There are several possible ways of improving these extensions in the future. First, we consider the extension
providing improved asynchronous I/O facilities. One possible way of improving this extension would be to expand
the functionality beyond just a simple translation of \verb|spawn| to \verb|io_event_add| and \verb|await| to 
\verb|io_dispatch|. While this functionality does provide better syntax for this project, having improved syntax to allow
for more complicated constructs would be useful. One potential direction for this extension to go would be to follow a
similar pattern to the \verb|select| construct in Go. This allows for programmers to wait for multiple communication
operations to finish and then execute a set of instructions based on which communication finishes first. This, 
implemented as an extension, would allow programmers to deal with conditions where we want to perform a read or
write as soon as possible, as opposed to simply doing so asynchronously.

We now consider what further improvements could be made to our Wuffs extension. First, we note that this the parser
we have implemented in this project is relatively simple, only parsing text as input, where other parsers must parse
much more complicated entities like images, audio files, and other file formats. In a more expansive project, it would be
a nice feature for a programmer to be able to include in multiple locations different Wuffs blocks. To do this, we would
need to allow the programmer to specify the Wuffs package name, as well as what calls to Wuffs functions the
programmer would like to make. This would potentially eliminate some of the automation from the Wuffs extension,
but it would still provide the same advantages (just to a lesser extent) over placing the Wuffs code in a separate file that
must be integrated into the project separately.

\end{document}