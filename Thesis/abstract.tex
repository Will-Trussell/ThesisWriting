\documentclass[main.tex]{subfiles}

\begin{document}
\begin{center}\textbf{Abstract}\end{center}
Extensible languages, first introduced in the 1960s by McIlroy, offer a way of adding new syntax to a base language.
Various extensible languages have been introduced, but these languages have not been implemented in full-scale
software projects. This work utilizes an extensible version of C to develop and implement new extensions to improve
and modify an implementation of an Internet Relay Chat (IRC) server. Three extensions are introduced in order to
show that extensions can be useful and effective in simplifying the process of developing code used in modern
software applications. Two of these three extensions introduce syntax to allow for compile-time checking of features
that are not able to be checked at compile-time in C, while the third introduces new syntax to provide programmers
with easier and more concise syntax for performing asynchronous I/O. This work not only examines the benefits
provided by these extensions, but also includes an examination of the drawbacks of these extensions, if such
drawbacks do exist. This work demonstrates that the benefits of extensible programming are real and extensions,
particularly those including compile-time checking of concepts not available in vanilla C, offer significant improvements
in code writing and readability over the same software written in plain C.
\end{document}