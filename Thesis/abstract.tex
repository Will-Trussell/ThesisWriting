\documentclass[main.tex]{subfiles}

\begin{document}
\begin{center}\textbf{Abstract}\end{center}
Extensible languages, first introduced in the 1960s by McIlroy, offer a way of adding new syntax and semantics to a
base language. This work utilizes \verb|ABLE|C, an extensible version of C conforming to the C11 standard, to develop 
and implement new extensions to \verb|ABLE|C to improve
and modify an implementation of an Internet Relay Chat (IRC) server. Two extensions are developed and a third is utilized in order to
qualitatively show that extensions can be useful and effective in simplifying the process of developing code used in modern
software applications. Two of these three extensions introduce syntax to allow for compile-time checking of features
that cannot be checked at compile-time in C, while the third introduces new syntax to provide programmers
with easier and more concise way of performing asynchronous I/O. Of the three extensions, two were written 
specifically for this work, while the third extension used already existed and was applied for this work. This
work not only examines the benefits provided by these extensions, but also includes an examination of the drawbacks
of the extension allowing for asynchronous I/O. This work demonstrates via qualitative comparisons that the
benefits of extensible programming are real and extensions, particularly those including compile-time checking of
concepts not available in vanilla C, offer significant improvements in code writing and readability over the same
software written in plain C.
\end{document}