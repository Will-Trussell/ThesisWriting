\documentclass[main.tex]{subfiles}

\begin{document}

\section{Wuffs Parsing}

\subsection{Wuffs Overview}
A final extension implemented for this project involves the Wuffs language.
Wuffs is a language created by Google for memory-safe handling of untrusted file
formats. Originally intended for use in security-specific portions of a program,
Wuffs allows for compile-time checking of various security vulnerabilities, as
well as some semi-rigorous proofs of program safety.

A simple parsing program in C is shown below. The same parsing program is also
shown in Wuffs as a comparison of the two languages. Both of these programs take
as input a string of numbers, for example, \verb|"123"|, and return the
associated numeric value, in this case, 123.

\begin{singlespace}
    \begin{figure}[p!]
        \lstinputlisting{./res/parse_bg.c}
        \caption{Parsing in C}
    \end{figure}
    \begin{figure}[p!]
        \lstinputlisting{./res/parse_bg.wuffs}
        \caption{Parsing in Wuffs}
    \end{figure}
\end{singlespace}

Leaving aside the syntactic peculiarities of Wuffs, note that the Wuffs code is
signficantly more verbose than the C code in the above examples. However, we
consider the edge cases of parsing some string, particularly with regards to
integer overflow. In C, an integer overflow error can occur, but it does so
silently. Obviously, in a real C parser implementation, we would implement
security checks to cover these overflow cases. The point, however, is less about
the fact that this is possible in C and more about the fact that these checks
are mandatory in Wuffs. The Wuffs compiler will not allow an addition expression
that could possibly overflow the maximum integer value.

This compile-time security checking for basic security vulnerabilites is
incredibly useful, particularly in contexts where we are parsing unknown code or
strings that could be malicious. We don't need to worry about runtime security
checks when we can use Wuffs to guarantee no buffer overflows, out-of-bounds 
array accessing, integer overflow, or integer underflow at compile time.

\subsection{Wuffs Extension}
In this section, we discuss the mechanics of the Wuffs extension. For more details, examine the concrete syntax file
for the \verb|ABLE|C-wuffs extension. The extension is surprisingly simple, given the benefits it adds to a project.

The extension simply looks for a block of code delineated by the markers \verb|WUFFS| and \verb|WUFFS_END|. The
code within these two markers is assumed to be valid Wuffs code. This code is loaded into its own \verb|.wuffs| file,
which is then compiled using the Wuffs compiler into a C file. In the original \verb|ABLE|C file, we replace the original
block of Wuffs code with boilerplate code that includes linking the new C file with the Wuffs functions, as well as calling
those functions to parse the input, store the input, and validate the results. Note in the code snippet of the parse 
function using the extension that all of the Wuffs code is within the larger \verb|parse()| function. This means that we
need to do minimal editing of the rest of the file. We can retain the already written functions provided out-of-the-box by
the server. In fact, in this case, the only function from the \verb|parse.c| file that needs to be modified is the primary
\verb|parse()| function. The end programmer does not need to concern themselves with writing the C code to call any
Wuffs functions -- the extension is able to take care of inserting those function calls for us, allowing the programmer to
simply write the parsing details in Wuffs.

\subsection{Wuffs Implementation}
Below, we have two code snippets of a parsing function. The first snippet is from the original server code.
It contains a C implementation of the parser. The second code snippet is an edited Wuffs implementation of the parser, with some code removed for brevity. 
\begin{figure}[hp]
	\lstinputlisting{./res/parse_before.c}
	\caption{The parse function before adding the Wuffs extension}
\end{figure}
\begin{figure}[hp]
	\lstinputlisting{./res/parse_after.c}
	\caption{The parse function after adding the Wuffs extension}
\end{figure}

We consider the differences between these two implementations. Obviously, there is a difference in syntax
between the two languages. Beyond that, however, we consider the differences between the two implementations.
First, we must primarily note that the C implementation has no guarantees of safety. If the program compiles, all
we can be sure of is that there are no syntax errors in the file. We have no guarantees that, given some malicious
input string, we won't be vulnerable to any attacks. In our Wuffs implementation, this code is sent off to the Wuffs
compiler during compilation. The Wuffs compiler is able to tell us whether we are vulnerable to buffer overflows or
other attacks. This is the largest benefit of using this extension. We get the speed of C code, but the safety and
guarantees provided by the Wuffs compiler.

\end{document}
