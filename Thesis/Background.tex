\documentclass[main.tex]{subfiles}

\begin{document}

\section{Background}

This chapter provides background information for the work that follows
throughout the paper. We first discuss the Silver attribute grammar system 
\cite{3}, which is used to implement the extensions to the host language.
We follow this with a discussion of \verb|ABLE|C \cite{9}, the extensible
version of C utilized in this work.

\subsection{Silver} %2.1
Silver \cite{3} is an extensible attribute grammar specification system that allows us to 
add both general features (pattern matching, for instance) and domain-specific features
to the host language. This gives us an attribute grammar specification system with a rich
set of language features that we utilize in defining the syntax and semantics of any new 
languages or extensions we may write.

Silver has several nice features useful in generating new language extensions.
First and foremost, Silver allows for \emph{forwarding} \cite{4} to 
implement new extensions. Forwarding allows language
designers to translate new syntax into the original host language. Using forwarding,
language designers no longer need to explicitly define semantic analysis for each new
piece of syntax in a language extension. Instead, any semantic analyses not defined
by the language designer are instead defined via translation. A brief example of forwarding is 
given here; for a more in-depth discussion, see the examples given in Section 1 of the paper 
\emph{Forwarding in Attribute Grammars for Modular Language Design} \cite{4}.

As an example, consider that we have some language with a defined string concatenation operator $++$.
Now consider adding an extension to the language, the operator $++_S$. We desire this operator to do
the following: given the expression $s_1 {++}_S s_2$, we would like the resulting expression to be $s_2$
\emph{sandwiched} between $s_1$. Thus, we would like the result of $s_1 {++}_S s_2$ to be $s_1 s_2 s_1$.
We could fully define the semantics of the new ${++}_S$ operator, but forwarding allows us to save both time and
effort. Instead, we can simply define our new operator as follows.
\begin{lstlisting}
sandwich : str_0 ::= str_1 str_2
	sandwich.errors = case str_1, str_2 of
		| String, String -> None
		| _, _ -> Error("Expected two strings, instead got" ++ 
				str_1.type ++ "and" ++ str_2.type
		end;
	forwards to concat(str_1, concat(str_2, str_1));
\end{lstlisting}
By doing this, we are able to quickly define a new piece of syntax for our language, allowing us to add new
syntax and semantics to a language quickly and easily. We still are able to define new semantics, like with the
"errors" attribute shown above, but we do not need to fully define the translation of our new operator. Instead,
that translation is taken care of by the already existing concat production.

\subsection{Extensible Programming and Able-C} %2.2
One of the primary programming languages that is utilized in modern computing
when speed or low-level control is vitally important is the C programming
language. Unfortunately, C lacks many of the features of more modern programming
languages, often making it cumbersome to work with in certain applications.

\verb|ABLE|C \cite{9} is a project that tries to improve upon the C language is through the use
of extensions. Built utilizing Silver, \verb|ABLE|C is an 
extensible C pre-processor, conforming to the C11 standard \cite{9}. It takes an
"extended" version of C (in a \verb|.xc| file), checks the file for errors, and then translates that
file back into plain C, performing
transformations and analyses as it does so. For instance, there is an existing %Cite sqlite?
\verb|ABLE|C extension that enables interaction with SQLite databases. With this
extension, we begin with the \verb|ABLE|C files. These files are then translated
into plain C code after going through type-checking and error-checking by the
\verb|ABLE|C compiler. %Need to add more detail in the example?
Another extension that provides a good example of the various analyses done by
\verb|ABLE|C is the \verb|ABLE|C-refcount-closure extension. This extension
implements reference counting of closures in C. The function performs analysis
on how many references exist to a given closure to help with memory management.
This is not a feature that is available in plain C code, but it is a useful tool for
programmers to have in \verb|ABLE|C.

\end{document}
