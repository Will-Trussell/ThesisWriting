\documentclass[main.tex]{subfiles}

\begin{document}

\section{Background}

This chapter provides background information for the work that follows
throughout the paper. We first discuss the Silver attribute grammar system 
\cite{silver}, which is used to implement the extensions to the host language.
We follow this with a discussion of \verb|Able|C \cite{ableC}, the extensible
version of C utilized in this work.

\subsection{Silver}
Silver \cite{1}, created by Van Wyk et al. is an attribute grammar
specification system. Furthermore, Silver is extensible, allowing us to add both
general features (pattern matching, for instance) and domain-specific features
to Silver. This gives us an attribute grammar specification system with a rich
set of language features we can utilize in developing new extensions. 

Silver has several nice features useful in generating new language extensions.
First and foremost, Silver allows for \emph{forwarding} \cite{forwarding} to 
implement new extensions in cost-effective ways. Forwarding allows language
designers to utilize some form of inheritance within their language extensions,
saving designers significant time and effort in creating new language features.

\subsection{Extensible Programming and Able-C}
One of the primary programming languages that is utilized in modern computing
when speed or low-level control is vitally important is the C programming
language. Unfortunately, C lacks many of the features of more modern programming
languages, often making it cumbersome to work with in certain applications.

One way that some have tried to improve upon the C language is through the use
of extensions. Writing extensions to the C language, however, can be quite
difficult, often involving many complications. For instance, the Cilk extension
was introduced to C to allow for easier parallel programming. However, the
original implementation of Cilk5 utilized its own type-checker, despite not
changing the underlying C type system (CITE Aaron's thesis, specifically the 
section at the bottom of page 14). 

This difficulty in extending the C language was one of the motivations behind
creating the \verb|ABLE|C language. Built utilizing Silver, \verb|ABLE|C is an 
extensible C pre-processor, conforming to the C11 standard \cite{ableC}. It takes an
"extended" version of C and translates it back into plain C, performing
transformations and analyses as it does so.

\end{document}
