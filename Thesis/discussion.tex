\documentclass[main.tex]{subfiles}

\begin{document}

\section{Conclusion}
We have seen several extensions implemented within the example of an IRC Server written in C. Previously, we have been
mostly focused on simply examining the differences between the original and modified servers, without passing judgement on
whether those differences are benefits or drawbacks. Within this section, we will focus more on the benefits and drawbacks of
the implemented extensions. We finish with a brief discussion of whether these extensions are realistically applicable in a
real-world piece of software.

\subsection{Non-null Pointers}
Consider the extension implementing compile-time and runtime checking for null pointer dereferences, discussed above in section 3.
This extension, while relatively simple in its final form, is actually incredibly useful. When compared with a C implementation of
a function that takes pointers as arguments, this extension offers immense upside. The programmer no longer needs to worry
about checking every pointer for null dereferences at the beginning of the function.

One additional benefit of this extension is the possibility of guaranteeing that the pointer returned from a function is not null.
Again, this would allow the programmer to worry much less about checking that pointers are safely dereferenced. This would
be particularly useful in contexts where we are not easily able to restart a program if it aborts due to a null pointer 
dereference; for instance, a singular server running communication between multiple clients (like the IRC server example
discussed throughout this paper) is not easily able to restart if it is forced to abort due to an unexpected null pointer
dereference. If this were to happen, all of the data stored on the server (messages, channels, and users, as well as their
associated histories) would be lost.

There are no readily apparent drawbacks to using our non-null extension, except for the idea that this extension likely will
result in minor slowdowns to code. While not tested using any sort of reliable benchmark, consider the example illustrated
in Figures 1, 2, 3, and 4. In particular, we consider Figures 1 and 4. These figures contain the C code that is being compiled and
run by the host machine. We observe that Figure 1 contains 6 executable lines of code. Figure 4, on the other hand, contains
significantly more lines of code that can be run. This would seem to indicate, at least upon a cursory examination, that the code in
Figure 4 would run slightly slower, simply because there are more instructions to execute, though the difference in speed would
likely be negligible in this application.

\subsection{Asynchronous I/O}
Next, we consider the extension implementing nicer facilities for asynchronous I/O than the syntax provided by plain C code.
First, we must note that this extension provides only repackaged syntax, not new functionality, when compared with the
\verb|epoll| API. The one major benefit provided by this extension, however, is the increase in readability of the code. We no
longer must deal directly with the intricacies of the \verb|epoll| API and the concept of \emph{events}. Instead, the extension
allows us to focus on the conceptual ideas in the code; that is, we simply want some function task to be run asynchronously,
and we don't need to worry about modifying which tasks the \verb|epoll| instance is tracking manually.

One drawback to our asynchronous extension is that we potentially lose some of the finer grains of control provided to us
by the raw \verb|epoll| API. As an example, consider Figure 5, specifically line 45. In this code pulled from the original IRC server, we see
a call to \verb|io_event_add| with the constant \verb|IO_WANTREAD|. This indicates that we are starting an asynchronous task that specifically
wants to perform \verb|read| operations in I/O. Consider the version of this code with the extension implemented in Figure 6, again at line
45. Here, we no longer have the ease of simply specifying that we want to perform an asynchronous \verb|read| operation. Instead, we must
define the specific function we want to run.

While this lack of control could potentially be improved using a more complex extension, we did
not consider such an extension within this paper. Additionally, adding complexity to the extension is perhaps antithetical to the
idea of extensions in the first place. We wish to abstract away certain aspects of the host language using new syntax,
so while we can achieve a more full control using more complexity, this added complexity may become more confusing than
simply using the original host language to achieve the same effect.

\subsection{Wuffs}
Finally, we consider the Wuffs extension that allows for more secure parsing. This extension is a wonderful example of
the benefits of using extensible programming languages. We introduce a secure way of parsing using the Wuffs language,
leveraging this new tool to write a secure parsing function. The benefits to using this extension are vast, especially when
compared with the original C parsing function. The guarantees offered by Wuffs allow us to be certain that our program has
no vulnerabilities to a variety of attacks. Additionally, with this extension, we no longer must worry about manually writing out
the code to make the necessary calls to the Wuffs code after it has been compiled into C. Instead, the extension is able to
take care of inserting these function calls as necessary. Another benefit of this extension is that we do not need to worry about
integrating a separate Wuffs file. We are instead able to write the Wuffs code \emph{in the same file} as the C code that will interact with
the Wuffs code.

Even with the added guarantees provided by the Wuffs compiler (requiring more checks than would be required in C code),
the Wuffs extension will not have a noticeable slowdown, and could even offer a moderate speed-up in certain cases. For
more details, see the benchmark documentation in the Wuffs repository for a comparison of various Wuffs libraries and
default C implementations of those same libraries \cite{5}. 

\subsection{Applications in Other Software}
We have seen the benefits brought by the extensions introduced in this paper. But are extensions a realistic way of writing
software for any real-world application? The work within this paper suggests that extensions do indeed provide concrete
benefits in real-world applications. We have seen that extensions, particularly those introducing compile-time checking of
various language features, are incredibly useful in saving programmers time and energy when writing new software. It is not
unreasonable to assume that an extension similar to the non-null extension discussed in this paper would be incredibly useful
in a large-scale C project.

Additionally, extensions similar to the Wuffs extension would be incredibly useful in scenarios where we are dealing with
safety-critical pieces of code. Any code that deals with possibly malicious inputs would benefit from the guarantees provided
by Wuffs. From buffer overflows to integer overflows or underflows, the Wuffs extension guarantees that we are protected from
these attacks, allowing programmers to no longer worry about writing more complicated safety checks in C.

\end{document}
