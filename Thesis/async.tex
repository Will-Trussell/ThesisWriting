\documentclass[main.tex]{subfiles}

\begin{document}

\section{Asynchronous I/O}

\subsection{Asynchronous I/O}
Asynchronous I/O is a major part of the difficulty in creating a server
implemenation in C, particularly when using threads is not a viable option. We
don't want the server to be stuck while we wait for it to perform some I/O
operation like reading or writing to an existing connection when the server has
other tasks it could be doing, like establishing new client connections or
parsing an already-received message. In C, one of the primary ways of performing
asynchronous I/O is using the \verb|epoll| API. Using this API, we are able to
keep a list of file descriptors we want the current process to monitor, as well
as a list of file descriptors that are ready for I/O. However, this process is
extraordinarily tedious, requiring many expensive system calls to set up and
maintain the \verb|epoll| instance. When compared with other, more modern
languages, the \verb|epoll| API is both more verbose and more difficult to use.
Consider a more modern language like Go or Javascript. Both of these languages
have their own facilities for asynchronous operation. In Go, we use several
constructs, including the \verb|go| and \verb|select| keywords, to implement
various aspects of I/O. In Javascript, we utilize both the \verb|async| and
\verb|await| keywords, as well as the idea of \emph{promises} in order to
achieve some measure of asynchronous operation. The goal of our extension is to
include these same facilities in \verb|ABLE|C, allowing for a programmer to more
easily write and understand the code performing asynchronous I/O operations.

\subsection{Asynchronous I/O Extension}
%Discussion of extension, mechanics behind it
In this section, we discuss the mechanics of the Asynchronous I/O extension, including
the specifics of the translation from \verb|ABLE|C code (\verb|.xc| files) to plain C code.

\subsection{Asynchronous I/O Implementation}
%Discussion of actual implementation of extension

\begin{figure}
\end{figure}
\begin{figure}
\end{figure}

\end{document}