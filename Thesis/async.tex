\documentclass[main.tex]{subfiles}

\begin{document}

\section{Asynchronous I/O}

\subsection{Asynchronous I/O}
Asynchronous I/O is a major part of the difficulty in creating a server
implemenation in C, particularly when using threads is not a viable option. We
don't want the server to be stuck while we wait for it to perform some I/O
operation like reading or writing to an existing connection when the server has
other tasks it could be doing, like establishing new client connections or
parsing an already-received message. In C, one of the primary ways of performing
asynchronous I/O is using the \verb|epoll| API. Using this API, we are able to
keep a list of file descriptors we want the current process to monitor, as well
as a list of file descriptors that are ready for I/O. However, this process is
extraordinarily tedious, requiring many expensive system calls to set up and
maintain the \verb|epoll| instance. When compared with other, more modern
languages, the \verb|epoll| API is both more verbose and more difficult to use.
Consider a more modern language like Go or Javascript. Both of these languages
have their own facilities for asynchronous operation. In Go, we use several
constructs, including the \verb|go| and \verb|select| keywords, to implement
various aspects of I/O. In Javascript, we utilize both the \verb|async| and
\verb|await| keywords, as well as the idea of \emph{promises} in order to
achieve some measure of asynchronous operation. The goal of our extension is to
include these same facilities in \verb|ABLE|C, allowing for a programmer to more
easily write and understand the code performing asynchronous I/O operations.

\subsection{Asynchronous I/O Extension}
%Discussion of extension, mechanics behind it
In this section, we discuss the mechanics of the Asynchronous I/O extension, including
the specifics of the translation from \verb|ABLE|C code (\verb|.xc| files) to plain C code.

This extension is less focused on redesigning the fundamentals of asynchronous I/O and is more focused
on providing syntax that is both easy to understand and easy to write for a programmer utilizing the extension.
To this end, we retain some of the same principles utilized by the \verb|epoll| API, but introduce similar syntax to
modern languages with the \verb|spawn| and \verb|await| keywords.

The \verb|spawn| keyword has similar syntax to the \verb|spawn| keyword in Cilk (cite Aaron's thesis? Or cilk docs),
the \verb|spawn()| method in Ruby, or the \verb|spawn()| method in Rust. All of these keywords have a similar idea
that underlies them. Each of these languages uses \verb|spawn| to indicate starting a new process, thread or function call.
The syntax for all three is also similar in concept, as well. Each has the form \verb|spawn <foo>|, where what is in 
\verb|foo| is either a function or a closure (in Rust). This starts program execution on whatever task is passed through 
\verb|foo|.

The \verb|await| keyword is at least partially inspired by the syntax of Javascript. After we spawn several tasks using
\verb|spawn|, we can then specify that we would like to wait for those tasks using \verb|await|. The syntax is similar to
Javascript: we simply write \verb|await <foo>|, where \verb|foo|, after the \verb|await| call, refers to a task that was created and
run using the \verb|spawn| keyword. In other words, \verb|await| simply waits until at least one of the tasks we have spawned
finishes, then loads that task (or tasks) into \verb|foo|.

\subsection{Asynchronous I/O Implementation}
%Discussion of actual implementation of extension
Here we consider what changes are actually made by our extension. Consider the code snippets from the \verb|io.c| file, shown below.

Note in the first example that we utilize the default \verb|io_event_add()| and \verb|io_dispatch()| functions. A particular
issue that arises when dealing with these functions is the lack of high-level transparency in their calls. Perhaps 
\verb|io_event_add()| is somewhat clear, but the \verb|io_dispatch()| function is particularly opaque. We make a call to the
function with some \verb|timeval|, but we cannot determine the use of this function without substantial effort on the part of the
programmer. 

On the other hand, consider the implementation using the asynchronous extension. We are using the
\verb|spawn| keyword to create new tasks, in this case, calling out to a \verb|read_helper()| or \verb|write_helper()| function.
We do not require any knowledge of constants like \verb|IO_WANTREAD| or \verb|IO_WANTWRITE|, instead simply requiring
the programmer to pass in whatever function the programmer desires to run asynchronously. 

This extension is also a much more flexible implementation than what is available with the \verb|epoll| API. As an example,
consider if a programmer wanted to modify the server to do something other than just read or write data using the \verb|epoll|
API. In the old C code, this would require defining a set of new constants and substantial code modification. In our new
extension, all the programmer must be concerned with is writing whatever function they desire to be executed asynchronously.
Then, utilize \verb|spawn| with that function to register that function for asynchronous execution. There is no need to define
any new constants or for the end programmer to do any substantial coding (outside of the function they desire to run
asynchronously, which would have been written anyways without the extension).

\begin{figure}
	\lstinputlisting[language=C, firstline=10,basicstyle=\footnotesize]{./res/async_before.c}
	\caption{A file utilizing the asynchronous I/O Interface before implementing the extension}
\end{figure}
\begin{figure}
	\lstinputlisting[language=C,firstline=10,basicstyle=\footnotesize]{./res/async_after.c}
	\caption{A file utilizing the asynchronous I/O interface after implementing the extension}
\end{figure}

\end{document}