\documentclass[main.tex]{subfiles}

\author{Will Trussell}
\title{Introduction}

\begin{document}
\section{Introduction}

Extensible programming languages were first discussed in the 1960s. As
originally discussed in Douglas McIlroy's 1960 paper, the original concept was
to use a small number of macros in order to extend a compiler to accept very
general extensions to the original language accepted by the compiler 
\cite{McIlroy}. This has been modernized in various ways, including the idea
of extensible syntax, compilers, and even extensible runtimes (CITE? wiki has
dead link).

One example of an extensible programming language is \verb|ABLE|C \cite{ableC}, 
an extensible version of the C programming language, built on top of the 
\verb|Silver| \cite{silver} attribute grammar system. Various extensions exist for 
\verb|ABLE|C that were created to solve various problems. 

This paper examines the uses of extensible programming languages in a real-world
application: an Internet Relay Chat (IRC) server implementation. The IRC
protocol was originally designed for text-based conferencing \cite{IRC}, and
many server implementations are available. For this paper, we consider the
specific implementation of \verb|ngIRCd| (next-generation IRC daemon), written 
in the C programming language. There are several potential issues with this 
server implementation. First, any C program must include runtime checks to 
ensure that all pointers are not null. We introduce an \verb|ABLE|C extension to 
allow for compile-time checking of possible null pointer dereferences.  Another 
issue with the server is writing easily understandable I/O code in C. While the 
\verb|epoll| API provided by the Linux kernel allows for some asynchronous I/O, 
we introduce another \verb|ABLE|C extension to provide improved syntax for 
asynchronous I/O, akin to what Javascript provides. Finally, we introduce an 
extension to ensure that parsing code is secure. We do this utilizing the Wuffs 
programming language \cite{wuffs}, which allows for compile-time checking for various 
parsing attacks which must be manually accounted for at run-time in C.

\end{document}
