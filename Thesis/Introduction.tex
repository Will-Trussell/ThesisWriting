\documentclass[main.tex]{subfiles}

\author{Will Trussell}
\title{Introduction}

\begin{document}
\section{Introduction}

Extensible programming languages were first discussed in the 1960s. As
originally discussed in Douglas McIlroy's 1960 paper, the original approach was
to use a small number of macros in order to extend a compiler to accept any
general extensions to the original language accepted by the compiler 
\cite{7}. This idea has been modernized in various ways, including the idea
of extensible syntax and compilers that are themselves extensible.

One example of an extensible programming language is \verb|ABLE|C \cite{10}, 
an extensible version of the C programming language, built on top of the 
\verb|Silver| \cite{3} attribute grammar system. Various extensions exist for 
\verb|ABLE|C that were created to solve various problems or add features. Such
extensions include facilities for greater ease of writing parallel code in C, using syntax
similar to Cilk\cite{2}, an implementation of closures in C, and the addition of algebraic
data types and pattern matching.

This paper examines the uses of extensible programming languages when applied to a real-world
application: an Internet Relay Chat (IRC) server implementation. The IRC
protocol was designed for text-based conferencing \cite{8}, and
many server implementations are available. For this paper, we consider the
specific implementation of \verb|ngIRCd| (next-generation IRC daemon), written 
in the C programming language \cite{1}. There are several potential issues with this 
server implementation that present themselves to programmers modifying the server. 
First, any C program must include runtime checks to 
ensure that all dereferenced pointers are not null. We apply a pre-existing \verb|ABLE|C extension to 
allow for both compile-time and runtime checking for possible null pointer dereferences.  Another 
issue with the \verb|ngIRCd| server is writing easily understandable I/O code in C. While the 
\verb|epoll| API \cite{5} provided by the Linux kernel allows for asynchronous I/O, 
we introduce a new \verb|ABLE|C extension to provide improved and simpler syntax for 
asynchronous I/O, akin to what Javascript provides. Finally, we introduce a new 
extension to ensure that parsing code is secure from buffer overflow attacks. We do this utilizing the Wuffs 
programming language \cite{6}, which allows for compile-time checking for various 
parsing attacks which must be manually accounted for at run-time in C.

\end{document}
