\documentclass[12pt]{article}

\author{Will Trussell}
\title{Introduction}

\begin{document}
\section{Introduction}

Extensible programming languages (cite Douglas McIlroy) were first discussed in
the 1960s. The original basic concept was that you have a base language that can
be extended through some meta-language. This has been modernized to include 
various other concepts, including extensible syntax, extensible compilers, 
extensible runtimes, as well as the idea of separting content from form.

One example of an extensible programming language is \verb|AbleC|(cite), an 
extensible version of the C programming language, built on top of the 
\verb|Silver| (cite) attribute grammar system. Various extensions exist for 
\verb|AbleC| that were created to solve various problems.

This paper examines the uses of extensible programming languages in a real-world
application: an Internet Relay Chat (IRC) server implementation. The IRC
protocol was originally designed for text-based conferencing (cite IRC RFC).

Internet Relay Chat (IRC) is a protocol designed for text-based conferencing. It
is designed as a network of servers, users, and channels (reference to original
IRC RFC here from Oikarinen \& Reed). Within this paper, we will

Introduction should cover:
\begin{itemize}
    \item Overview of what the objective of the paper is
    \item Discussion of AbleC (reference to original papers)
    \item Discuss the original NGIRCd server
    \item Discuss motivation of adding these extensions
    \item At this point, probably want to discuss the specific extensions that 
        were implemented
        \begin{itemize}
            \item Discussion of nonnull can be fairly brief
            \item Discussion of Wuffs parsing
            \item Discussion of asynchronous
        \end{itemize}
    \item The discussions should include information on what motivates the 
        specific extensions.
\end{itemize}
\end{document}
