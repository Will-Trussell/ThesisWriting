\documentclass[main.tex]{subfiles}

\begin{document}

\section{Non-Null Pointer Extension}

\subsection{Non-null Qualifiers}
One of the most common class of bugs and errors in the C programming language is errors
related to null pointers. The dereferencing of a null pointer is considered
undefined behavior in C, and is undesirable when programming. In this section, we introduce an \verb|ABLE|C extension,
first created by Travis Carlson \cite{9}, to
deal with null pointers (referred to herein as \verb|ABLE|C-nonnull). This extension allows for compile-time
checking for any possible dereferences of null pointers. This static analysis of a program is valuable, saving
the programmer both time and effort when compared with the original C code, which contains only dynamic checks
at runtime that must be manually inserted by the programmer. 

\subsection{Comparison}

Consider the code snippets from \verb|ngIRC| presented in Figures 1 and 2. Both code snippets are from the same function that copies 
an array from one pointer to another in the \verb|array.c| (or \verb|array.xc|) file,
with Figure 2 having been rewritten using \verb|ABLE|C-nonnull. Note the differences between the two figures, namely, that we can
remove the dynamic checks that are present in Figure 1 after applying our extension.

\begin{figure}[h]
\lstinputlisting[language=C, firstline=84, lastline=96,basicstyle=\small]{./res/nonnull_before.c}
\caption{Function before applying the non-null extension}
\end{figure}
\begin{figure}[h]
\lstinputlisting[language=C, firstline=82, lastline=89,basicstyle=\small]{./res/nonnull_after.c}
\caption{Function after applying the non-null extension}
\end{figure}

The code in Figure 1 is a simple function to copy a string into an array as it is written in the original, unmodified project. It must explicitly
check for null pointer dereferences using the \verb|assert| statements. Note that if this function is passed a null pointer for either
of the two pointer arguments, the \verb|assert| statement will fail, causing a call to the \verb|abort()| function, crashing our
program.

Compare this to Figure 2, which is the same function, but with the non-null extension. Here, we know that both pointer
arguments are guaranteed to be non-null in the body of this function, as checks are inserted at the location of the function call. 
Thus, we no longer need either the assert statements or the check in
the \verb|if| statement. This is a toy example implemented to illustrate the point that including this extension allows for shorter 
functions and fewer manual safety checks on the part of the end programmer, though the \verb|nonnull| qualifier must still be inserted
in order for this extension to function.

As a final example of how this non-null extension works, we examine the translation of another function within the \verb|array.xc|
file. The purpose of this function is to append a trailing null byte to an array, but not count that byte in the length of the array.
Note that in Figure 3, we are casting the results of a call to \verb|array_alloc|. This is something the non-null extension will check
at compile time. When we examine Figure 4, we see that the same cast is present, but there is a runtime check inserted after
the cast to ensure that the pointer is not null. This is one way the extension guarantees that pointer dereferences are non-null. If the
dereferences are null, the programmer will discover this immediately, allowing for an immediate debugging, instead of only finding out
about the null pointer after trying to dereference the pointer in another code location.
Additionally, note that the translated C code in Figure 4 contains many more parenthesis than would be strictly necessary.
These extraneous parenthesis are inserted during the translation process, though the translated code is not necessarily intended to
be examined by programmers.
\begin{figure}[p]
\lstinputlisting[language=C, lastline=6, basicstyle=\small]{./res/translation.c}
\caption{A function to append a null byte to an array}
\end{figure}
\begin{figure}[p]
\lstinputlisting[language=C, firstline=8, basicstyle=\small]{./res/translation.c}
\caption{The same function to append a null byte to an array after being translated to C}
\end{figure}


\end{document}