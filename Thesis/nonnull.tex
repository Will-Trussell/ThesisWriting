\documentclass[main.tex]{subfiles}

\begin{document}

\section{Non-Null Pointer Extension}

\subsection{Non-null Qualifiers}
One of the most common sources of bugs in the C programming language is errors
related to null pointers. The dereferencing of a null pointer is considered
undefined behavior in C, and is undesirable when programming. In this section, we introduce an \verb|ABLE|C extension to
deal with null pointers (referred to herein as \verb|ABLE|C-nonnull). This extension allows for compile-time
checking for any possible null dereferences of pointers. This static analysis of a program is valuable, saving
the programmer both time and effort when compared with the original C code, which contains only dynamic checks
at runtime. 

\subsection{Comparison}

Consider the code in Figures 1 and 2. Both code snippets are from the same function in the \verb|array.c| (or \verb|array.xc|) file,
with Figure 2 having been rewritten using \verb|ABLE|C-nonnull. Note the differences between the two figures.

\begin{figure}[h]
\lstinputlisting[language=C, firstline=84, lastline=96,basicstyle=\small]{./res/nonnull_before.c}
\caption{Function before applying the non-null extension}
\end{figure}
\begin{figure}[h]
\lstinputlisting[language=C, firstline=82, lastline=89,basicstyle=\small]{./res/nonnull_after.c}
\caption{Function after applying the non-null extension}
\end{figure}

The code in Figure 1 is a simple function to copy a string into an array as it is written in the original project. It must explicitly
check for null pointer dereferences using the \verb|assert| statements. Note that if this function is passed a null pointer for either
of the two pointer arguments, the \verb|assert| statement will fail, causing a call to the \verb|abort()| function, crashing our
program.

Compare this to Figure 2, which is the same function, but with the non-null extension. Here, we know that both pointer
arguments are guaranteed to be non-null. Thus, we no longer need either of the assert statements, nor do we need the check in
the \verb|if| statement. This is a toy example implemented to illustrate the point that including this extension allows for shorter functions and fewer safety checks on the part of the end programmer.

As a final example of how this non-null extension works, we examine the translation of another function within the \verb|array.xc|
file. The purpose of this function is to append a trailing null byte to an array, but not count that byte in the length of the array.
Note that in Figure 3, we are casting the results of a call to \verb|array_alloc|. This is something the non-null extension will check
at compile time. When we examine Figure 4, we see that the same cast is present, but there is a runtime check inserted after
the cast to ensure that the pointer is not null. This is one way the extension guarantees that pointer dereferences are non-null. 
\begin{figure}[p]
\lstinputlisting[language=C, lastline=8, basicstyle=\small]{./res/translation.c}
\caption{A function to append a null byte to an array}
\end{figure}
\begin{figure}[p]
\lstinputlisting[language=C, firstline=10, basicstyle=\small]{./res/translation.c}
\caption{The same function to append a null byte to an array after being translated to C}
\end{figure}


\end{document}