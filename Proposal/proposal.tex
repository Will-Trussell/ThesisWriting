\documentclass[12pt]{report}

\usepackage[margin=1.0in]{geometry}
\usepackage{comment}
\usepackage{setspace}
\usepackage{lipsum} %Filler text

\title{Thesis Proposal} %Working Title
\author{Will Trussell}

\begin{document}

\doublespacing

An extensible programming language is a programming language that allows programmers to add extensions, which detail new syntax in the programming language. Extensible languages offer many advantages over traditional programming languages. They allow programmers to adapt the programming language to a particular problems in powerful ways. One such example of an extensible programming language is \verb|AbleC|, an extensible version of the C programming language at the C11 standard (Van Wyk et al. 2017). This thesis will examine the impact of applying an extensible language to an open-source software project. In particular, this thesis will examine the impact of applying \verb|AbleC| to the source code of an open-source Internet Relay Chat (IRC) server that was originally written in C. Key components of the server will be rewritten in \verb|AbleC|, and the modified server will be compared to the original server. These comparisons will be both quantitative (ensuring similar or better code performance, as well as the total number of lines of code) and qualitative (code readability and clarity). The comparisons will be used to analyze how useful extensible languages are when applied to existing open-source software.

\end{document}